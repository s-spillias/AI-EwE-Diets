\section*{Supplementary Material}

\subsection*{Research Focus Specification}
The validation process was conducted with the research focus specification: ``Future of Seafood''. This focus was consistently applied across all three study regions (Northern Territory, South East Inshore, and South East Offshore) to ensure comparability of results.

\subsection*{AI Prompting Protocol}
The following prompt was used to generate ecosystem descriptions from each AI model:

\begin{verbatim}
Given a marine area bounded by these coordinates:
Latitude: [min_lat]° to [max_lat]°
Longitude: [min_lon]° to [max_lon]°

Please provide a detailed description of this marine area, including:
1. The general geographic location and any notable features
2. The type of marine environment (e.g., coastal, pelagic, reef)
3. Typical oceanographic conditions
4. Major habitat types present
5. Key ecological characteristics

Focus on aspects that would be relevant for ecosystem modeling.
\end{verbatim}

\subsection*{Grouping Template}
The following table presents the complete grouping template provided to the AI models. This template served as a reference for potential functional groups, though models were not restricted to these groups and could create new ones based on regional characteristics.

\begin{longtable}{p{0.3\textwidth}p{0.7\textwidth}}
\caption{Complete Functional Group Template} \\
\hline
\textbf{Group Name} & \textbf{Description} \\
\hline
\endfirsthead
\multicolumn{2}{c}{{\tablename} \thetable{} -- Continued} \\
\hline
\textbf{Group Name} & \textbf{Description} \\
\hline
\endhead
\hline
\multicolumn{2}{r}{{Continued on next page}} \\
\endfoot
\hline
\endlastfoot
Skates and rays & Bottom-dwelling cartilaginous fish that play a role in controlling benthic prey populations \\
Nearshore and smaller seabirds & Small gulls, terns etc that feed near shore (possibly include penguins here too) - avian predators that link marine and terrestrial ecosystems \\
Albatrosses & Large seabirds that forage exclusively at sea, feeding on marine prey (fishes, squids, gelatinous organisms) \\
Skuas and giant petrels & Large predatory seabirds that feed both at sea and on land, including predation on other birds \\
Fish-eating pinnipeds & Marine mammals (seals, sea lions) that primarily prey on fish in coastal and pelagic ecosystems \\
Invertebrate-eating pinnipeds & Marine mammals (particularly Antarctic seals) that primarily feed on krill and other invertebrates \\
Baleen whales & Large filter-feeding marine mammals that regulate zooplankton populations and contribute to nutrient cycling \\
Orcas & Apex predators that uniquely prey upon other top predators including marine mammals, sharks, and large fish \\
Sperm whales & Deep-diving cetaceans that primarily feed on deep-water squid and fish \\
Small toothed whales and dolphins & Smaller cetaceans that primarily feed on fish and squid in surface and mid-waters \\
Sea snakes & Marine reptiles that prey primarily on fish, particularly eels and fish eggs \\
Crocodiles & Large predatory reptiles in coastal and estuarine waters that prey on fish, birds, and mammals \\
Turtles & Herbivores and omnivores that breed on land \\
Planktivores & Small fishes that feed on plankton, crucial in transferring energy from plankton to larger predators \\
Flying fish & Epipelagic fish capable of gliding above the water surface, important prey for many predators \\
Remoras & Fish that form commensal relationships with larger marine animals, feeding on parasites and food scraps \\
Large oceanic piscivorous fish & Fish-eating predators in open ocean environments, mid-sized non-migratory species (e.g. barracuda) \\
Tuna and Billfish & Large oceanic predatory fish, highly mobile, often dive to feed deeper into the water column \\
Shelf small benthivores & Small bodied fish that feed on benthic organisms, playing a key role in benthic-pelagic coupling, live in shelf waters \\
Shelf demersal omnivorous fish & Medium sized demersal fish that feed on invertebrates as well as smaller fish, live in shelf waters \\
Shelf medium demersal piscivores & Medium sized demersal fish living near the bottom in shallow waters, often important in benthic food webs, feed on other fish primarily, live in shelf waters \\
Shelf large piscivores & Fish-eating predatory fishes found in various marine habitats, important in controlling prey fish populations \\
Herbivorous demersal fish & Bottom-associated fish that primarily feed on plants, important in controlling algal growth \\
Slope/deep water benthivores & Small to mid sized fish that feed on benthic organisms and live on the shelf or seamounts \\
Slope/deep demersal omnivorous fish & Medium sized demersal fish that feed on invertebrates as well as smaller fish, live in slope or seamount waters \\
Slope/deep medium demersal piscivores & Medium sized demersal fish that feed on other fish primarily, live in slope or seamount waters \\
Slope/deep large piscivores & Fish-eating predatory fishes found in various marine habitats in deeper water, live in slope or seamount waters \\
Migratory mesopelagic fish & Fish living in the mesopelagic zone, undertake diel vertical migration, important in energy transfer between depths \\
Non-migratory mesopelagic fish & Fish living in the mesopelagic zone, non-migratory species, important in energy transfer between depths \\
Reef sharks & Top predators in coral reef ecosystems, controlling fish populations and maintaining reef health \\
Pelagic sharks & Open-ocean predators that help regulate populations of fishes and squids \\
Demersal sharks & Bottom-dwelling sharks, including dogfishes, that control populations of fishes and invertebrates on and near the seafloor \\
Cephalopods & Intelligent mollusks like squid and octopus, important predators in many marine ecosystems \\
Hard corals & Reef-building colonial animals that create complex habitat structure through calcium carbonate deposition \\
Soft corals & Colonial animals that contribute to reef habitat complexity without building calcium carbonate structures \\
Sea anemones & Predatory anthozoans that can form symbiotic relationships with fish and crustaceans \\
Hydrothermal vent communities & Specialized organisms living around deep-sea vents, including chemosynthetic bacteria and associated fauna \\
Cold seep communities & Organisms adapted to methane and sulfide-rich environments on the seafloor \\
Deep-sea glass sponges & Filter-feeding animals that create complex deep-water habitats and are important in silicon cycling \\
Sea cucumbers & Deposit-feeding echinoderms important in sediment processing and bioturbation \\
Sea urchins & Herbivorous echinoderms that can control macroalgal abundance and affect reef structure \\
Crown-of-thorns starfish & Coral-eating sea stars that can significantly impact reef health during population outbreaks \\
Benthic filter feeders & Bottom-dwelling organisms that filter water for food, important in nutrient cycling and regulating water quality in various depths - bivalves, crinoids, sponges \\
Macrozoobenthos & Mobile large bottom-dwelling invertebrates in both shallow and deep waters, important in benthic food webs and bioturbation (predatory or omnivorous) \\
Benthic grazers & Bottom-dwelling organisms that graze on algae and detritus, influencing benthic community structure \\
Prawns & Small crustaceans that are important in benthic and pelagic food webs \\
Meiobenthos & Tiny bottom-dwelling organisms, important in sediment processes and as food for larger animals \\
Deposit feeders & Animals that feed on organic matter in sediments, important in nutrient cycling \\
Benthic infaunal carnivores & Predatory animals living within the seafloor sediments \\
Sedimentary Bacteria & Microscopic organisms crucial in nutrient cycling and the microbial loop in marine ecosystems \\
Large carnivorous zooplankton & Fish larvae, arrow worms and other large predatory zooplankton \\
Antarctic krill & Key species in Antarctic food webs, particularly important as prey for whales, seals, and seabirds \\
Ice-associated algae & Microalgae living within and on the underside of sea ice, important primary producers in polar regions \\
Ice-associated fauna & Specialized invertebrates living in association with sea ice, important in polar food webs \\
Mesozooplankton & Medium-sized zooplankton (200 µm to 2 cm) that feed on smaller plankton and serve as food for larger animals \\
Microzooplankton & Tiny zooplankton (20 µm to 200 µm) that graze on phytoplankton and bacteria, forming a crucial link in the microbial food web \\
Pelagic tunicates & Including larvaceans, salps, and pyrosomes, important in marine snow formation and carbon cycling \\
Jellyfish & Predatory gelatinous species \\
Diatoms & Larger phytoplankton (20 µm to 200 µm), silica dependent important primary producers in marine ecosystems \\
Dinoflagellates & Mixotrophic species (20 µm to 200 µm) that can switch between primary production and consumption as needed \\
Nanoplankton & Plankton ranging from 2 µm to 20 µm in size, including small algae and protozoans \\
Picoplankton & Plankton ranging from 0.2 µm to 2 µm in size, including both photosynthetic and heterotrophic organisms \\
Microalgae (microphytobenthos) & Microscopic algae that live on the seafloor or attached to other organisms \\
Pelagic bacteria & Watercolumn dwelling bacteria, consume marine snow amongst other things \\
Seagrass & Marine flowering plants that form important coastal habitats and nursery areas \\
Mangroves & Salt-tolerant trees forming critical coastal nursery habitats and protecting shorelines \\
Salt marsh plants & Coastal vegetation adapted to periodic flooding, important in nutrient cycling and shoreline protection \\
Macroalgae & Seaweeds of various sizes that provide habitat and food for many species, including both canopy and understory forms \\
Symbiotic zooxanthellae & Photosynthetic dinoflagellates living within coral and other marine invertebrates \\
Cleaner fish and shrimp & Species that remove parasites from other marine animals, important in reef health \\
Discards & Carrion and freshly discarded material from fisheries activities \\
Detritus & Labile components of natural death and waste \\
\end{longtable}
