Ecosystem-based fisheries management (EBFM) has emerged as a critical policy goal for ocean management agencies worldwide \citep{FAO2003, EuropeanCommission2013, NOAA2016}. The practical implementation of ecosystem approaches to management requires ecosystem modeling within the context of natural resource management processes \citep{Collie2016}. Among these approaches, mass-balance models that track biomass flows between producers, consumers, predators, and fisheries have become foundational tools for understanding ecosystem structure and function \citep{Christensen2004}.

Ecopath with Ecosim (EwE) represents one of the most widely used food web modeling approaches in marine ecosystems \citep{Christensen2004, Colleter2015}. The model explicitly incorporates trophic interactions among multiple species and functional groups while maintaining mass conservation constraints across the broader food web. This makes EwE particularly valuable for quantifying trade-offs arising from natural or anthropogenic perturbations and assessing cumulative impacts of multiple stressors on marine ecosystems \citep{Coll2015, Villasante2016}.

A critical aspect of ecosystem modeling is the validation of model components and outputs. As model complexity increases to reflect biological realism, there is an unavoidable concurrent increase in scientific uncertainty due to limited knowledge of functional relationships \citep{PlaganyiButterworth2004}. This is particularly relevant for species grouping decisions, which form the foundation of any ecosystem model's structure. The reliability of ecosystem models depends heavily on appropriate species grouping that accurately represents the ecological roles and trophic interactions within the system.

Recent advances in artificial intelligence and machine learning present new opportunities for automating and validating species grouping decisions in ecosystem models. However, these approaches require rigorous validation to ensure they capture meaningful ecological relationships. This validation is essential as grouping decisions can significantly impact model behavior and subsequent management recommendations \citep{Heymans2016, Link2010}.

The validation of AI-based species grouping must consider both the ecological principles of functional group formation and the geographic context in which these groups operate. Marine ecosystems exhibit significant spatial heterogeneity in their physical, chemical, and biological characteristics, which directly influences species distributions, interactions, and functional roles \citep{Longhurst2007}. A species that serves as a key predator in one region may play a different ecological role in another due to variations in habitat availability, prey distributions, or environmental conditions. Therefore, validating AI-generated functional groups across distinct geographic regions is crucial for several reasons:

1. It tests the AI's ability to recognize and account for regional differences in ecosystem structure and function
2. It ensures that species groupings reflect local ecological contexts rather than applying a one-size-fits-all approach
3. It validates the AI's capacity to incorporate region-specific environmental and oceanographic factors that influence species' ecological roles
4. It helps identify potential biases or limitations in the AI's understanding of how geographic variation affects trophic relationships and ecosystem dynamics

This analysis examines the reliability and consistency of automated species grouping across three distinct marine regions: the Northern Territory, South East Inshore, and South East Offshore. Our validation framework addresses three key aims:

1. To assess AI models' understanding of regional marine ecosystems through their ability to generate accurate and comprehensive ecosystem descriptions
2. To evaluate the consistency of functional group generation across different AI models and marine regions
3. To analyze the ecological validity of AI-generated functional groups by comparing them against established ecological principles and known regional characteristics

By leveraging multiple AI models and comparing their performance across different ecological contexts, we aim to assess the robustness of automated grouping approaches and their potential role in ecosystem-based management.
