\section{Introduction}

Ecosystem modeling is a critical tool for understanding and managing complex environments, with Ecopath with Ecosim (EwE) being a well-established framework used to model and predict marine ecosystems \citep{Christensen2004, Colleter2015}. EwE models provide quantitative insights into ecosystem structure and function, enabling researchers to assess cumulative impacts of multiple stressors and support ecosystem-based fisheries management (EBFM) decisions \citep{Coll2015, Villasante2016,Geary2020}. However, constructing these models presents significant challenges, particularly in constructing and parameterizing diet matrices that capture the complex web of trophic interactions within an ecosystem.

Traditional approaches to EwE model development rely heavily on extensive literature review, data collation and expert knowledge, which are time-consuming and resource-intensive \citep{Holden2024a}. The process of assembling diet matrices is particularly challenging, requiring synthesis of diverse data sources including field studies, literature reviews, and expert opinion. This creates a significant bottleneck in model development, especially when applying models to new geographical contexts \citep{Holden2024b,spilliasfuture}. Recent advances in artificial intelligence (AI) offer new opportunities to streamline the model development process. AI tools have demonstrated success in both knowledge/evidence synthesis tasks \citep{spillias2024human,keck2025extracting,spillias2024evaluating,Zheng2023}, ecological and environmental tasks \citep{Fernandes2024,Li2024,Chen2024} and modelling tasks \citep{Lapeyrolerie2022, Tuia2022, Karniadakis2021}, but their application to process-based ecosystem modeling remains limited. The key challenge lies in ensuring that AI-driven approaches can effectively synthesize available information while maintaining ecological validity.

This study presents a novel framework for assembling and synthesizing local and online resources to parameterize EwE diet matrices using AI. Our approach integrates multiple data sources, including global biodiversity databases, species interaction repositories, and locally-held unstructured or structured text, to automate key steps in model development. The framework employs LLMs to group species into functional units and estimate trophic interaction magnitudes. We validate this framework through three case studies representing distinct Australian marine ecosystems: the Northern Territory, South East shelf, and South East offshore regions (Figure \ref{fig:validation_regions}). These regions offer contrasting environmental conditions, species assemblages, and ecological dynamics, providing a robust test of the framework's adaptability and reliability.

The primary objectives of this study are to:

\begin{enumerate}
    \item Present a systematic, AI-assisted framework for assembling and parameterizing EwE diet matricesred
    \item Validate key steps in the AI decision-making process, including:
        \begin{itemize}
            \item Species grouping decisions and their ecological validity
            \item Resulting diet matrix values and their reliability
        \end{itemize}
    \item Assess the framework's applicability across different marine ecosystems
\end{enumerate}

By rigorously validating this AI-assisted approach across multiple regions, we aim to demonstrate its potential for accelerating ecosystem model development while maintaining scientific rigor. Recent work by \citet{Kuhn2024} emphasizes how machine learning applications in fisheries must span multiple scales, from genomics to ecosystem-level analyses, while maintaining interpretability and transparency in automated decision-making systems. This work contributes to the growing need for rapid, data-driven methodologies in ecology \citep{Kelling2009, Michener2012}, while ensuring their outputs align with established ecological principles and can effectively support ecosystem-based management decisions. Our validation approach is informed by emerging standardization efforts in environmental AI applications \citep{Guo2025}, addressing the critical need for rigorous evaluation frameworks in ecological modeling.
