\subsubsection{Species Identification}

We begin species identification by accepting a GeoJSON file that defines the study region boundaries. We selected the Ocean Biodiversity Information System \citep{Grassle1999} as our primary data source due to its extensive marine species coverage and standardized taxonomic classifications. We access OBIS through the \texttt{robis} R package \citep{robis}, which enables automated querying and data processing.

We process the GeoJSON file in two steps to ensure precise geographic filtering. First, we extract a bounding box from the GeoJSON geometry. We then convert this box into a polygon string for database querying. Through the OBIS checklist function, we retrieve scientific names and complete taxonomic classifications from kingdom to species level for all recorded occurrences within these boundaries.

We clean the raw occurrence data through three sequential steps. First, we filter the dataset using OBIS's \texttt{is\_marine} flag to eliminate terrestrial species that may occur in coastal records. Second, we remove taxonomic redundancy using a rank-based approach that retains only the most specific classification level available. Our algorithm processes taxonomic ranks from most specific (scientific name) to most general (kingdom), keeping only the entry with the highest taxonomic resolution for each organism. Third, we aggregate occurrences by unique species while preserving their complete taxonomic hierarchies, calculating occurrence frequencies for each species.

We store the final species list in a structured CSV file containing verified marine species, their complete taxonomic hierarchies, and occurrence frequencies. This standardized format facilitates efficient data transfer to subsequent framework stages while maintaining provenance information. The complete R implementation, including rank-based filtering algorithms and geographic processing functions, is available in the project repository.
