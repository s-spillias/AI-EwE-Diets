\subsubsection{Species Grouping}

We implement species grouping through a flexible system that adapts to regional ecological contexts. The framework supports three approaches for defining functional groups: using templates from existing EcoBase models, generating region-specific groups through AI analysis of geographic characteristics, or applying a predefined template (provided in Section~\ref{tab:functional_groups} of the supplementary material). For geographic regions, we analyze oceanographic conditions, habitat types, and ecological characteristics to inform group definitions.

We process taxa hierarchically from kingdom to species level, implementing an incremental system with progress tracking to handle large datasets reliably. At each taxonomic level, Claude evaluates taxa against the selected grouping template using the following prompt:

\begin{prompt}
You are classifying marine organisms into functional groups for an Ecopath with Ecosim (EwE) model. Functional groups can be individual species or groups of species that perform a similar function in the ecosystem, i.e.\ have approximately the same growth rates, consumption rates, diets, habitats, and predators. They should be based on species that occupy similar niches, rather than of similar taxonomic groups.

Examine these taxa at the [rank] level and assign each to an ecological functional group.

Rules for assignment:
\begin{itemize}
\item If a taxon contains members with different feeding strategies or trophic levels, assign it to `RESOLVE'
\item Examples requiring `RESOLVE':
  \begin{itemize}
  \item A phylum containing both filter feeders and predators
  \item An order with both herbivores and carnivores
  \item A class with species across multiple trophic levels
  \end{itemize}
\item If all members of a taxon share similar ecological roles, assign to an appropriate group
\item Only consider the adult phase of the organisms, larvae and juveniles will be organized separately
\item Only assign a definite group if you are confident ALL members of that taxon belong to that group
\end{itemize}

Taxa to classify:
[List of taxa]

Available ecological groups (name: description):
[List of available groups and their descriptions]

Return only a JSON object with taxa as keys and assigned groups as values.
\end{prompt}

When the research focus indicates areas requiring higher resolution (e.g., commercial fisheries species), we modify the classification process with additional guidance:

\begin{prompt}
Special consideration for research focus:
The model's research focus is: [research focus]

When classifying taxa that are related to this research focus:
\begin{itemize}
\item Consider creating more detailed, finer resolution groupings
\item Keep species of particular interest as individual functional groups
\item For taxa that interact significantly with the focal species/groups, maintain higher resolution groupings
\item For other taxa, broader functional groups may be appropriate
\end{itemize}
\end{prompt}

We process each taxon through this classification framework, generating structured JSON documents that map species to functional groups. Taxa marked as \texttt{RESOLVE} undergo evaluation at finer taxonomic levels until reaching a definitive group assignment or the species level. We maintain complete provenance information, including the source of group definitions and any AI-suggested modifications.
