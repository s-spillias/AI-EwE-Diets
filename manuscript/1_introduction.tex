\section{Introduction}

Ecosystem modelling is a critical tool for understanding and managing complex environments, with Ecopath with Ecosim (EwE) being a well-established framework with thousands of implementations used to model marine ecosystems and predict their responses to external pressures \citep{Christensen2004, Colleter2015}. EwE models provide quantitative insights into ecosystem structure and function, enabling researchers to assess cumulative impacts of multiple stressors and support ecosystem-based fisheries management (EBFM) decisions \citep{Coll2015, Villasante2016,Geary2020}. However, constructing these models presents significant challenges, particularly in constructing and parameterizing diet matrices that capture the complex web of trophic interactions within an ecosystem.

Traditional approaches to EwE model development rely heavily on extensive literature review, data collation and expert knowledge, which are time-consuming and resource-intensive \citep{Holden2024a}. The process of assembling diet matrices is particularly challenging, requiring synthesis of diverse data sources including field studies, literature reviews, and expert opinion. This creates a significant bottleneck in model development, especially when applying models to new geographical contexts \citep{Holden2024b}. Recent advances in artificial intelligence (AI) offer new opportunities to streamline the model development process and avoid such bottlenecks \citep{spilliasfuture}. AI tools have demonstrated success in both knowledge/evidence synthesis tasks \citep{spillias2024human,keck2025extracting,castro2024large,spillias2024evaluating,Zheng2023,nugraha2024traditional}, ecological and environmental tasks \citep{Fernandes2024,Li2024,Chen2024,dorm2025large,Noleto2024} and modelling tasks \citep{Lapeyrolerie2022, Tuia2022, Karniadakis2021}, but their application to process-based ecosystem modelling remains nascent. The key challenge lies in ensuring that AI-driven approaches can effectively synthesise available information while maintaining ecological validity.

We present a novel and flexible framework for assembling and synthesizing user-defined and online resources to parameterise EwE diet matrices using Large Language Models (LLMs). Our approach integrates multiple data sources, including global biodiversity databases, species interaction repositories, and locally-held unstructured or structured text, to automate key steps in model development. The framework employs user-selected LLMs to group species into functional units and estimate trophic interaction strengths. We evaluate the system in four distinct Australian marine ecosystems - the Northern Australia, South East shelf, and South East offshore regions, where we assess the reproducibility of the approach, and in the Great Australian Bight where we assess the accuracy of the approach. Specifically, we test the precision (repeatability) and scientific accuracy of automated species grouping decisions, and the precision and accuracy of the resulting diet matrix proportions, with accuracy defined in terms of similarity to expert estimates. These regions offer contrasting environmental conditions, species assemblages, and ecological dynamics, providing a robust test of the framework's adaptability and reliability.


