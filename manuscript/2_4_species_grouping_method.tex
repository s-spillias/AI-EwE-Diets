\subsubsection{Species Grouping}

We implemented a template-based approach to control the quality of functional groups that are defined by the LLM. The framework uses a user-defined grouping template (provided in Section~\ref{supp:4} of the supplementary material) that leverages ecosystem modelling experience while allowing for regional customization. Due to the complexity of defining ecological groups for EwE models, we have implemented additional template-generation options but do not use them for validation in this study (See the code repository for more details).

Because OBIS can return thousands of species for a given region, instead of using an LLM to classify each species individually, which is time- and cost-prohibitive, we group species hierarchically to reduce the number of classifications required. The framework does this iteratively traversing the resulting OBIS database, from kingdom to species, classifying taxonomic groups into functional groups at finer and finer resolutions. Starting at the Kingdom level, the LLM is asked to classify taxa into functional groups. Taxa that the LLM does not think fall neatly into a specific functional group undergo evaluation at finer taxonomic levels until reaching a definitive group assignment or the species level. 

For example, when classifying something like the Western Australian Dhufish (Glaucosoma hebraicum), after passing through the Kingdom Animalia, the phylum Chordata is evaluated. Since Chordata includes diverse feeding strategies from filter-feeding tunicates to predatory fish, the LLM marks it for resolution at a finer level. At the class level, the LLM evaluates Actinopterygii, which is again marked for resolution due to its diverse feeding strategies. Continuing through the taxonomic hierarchy, the family Glaucosomatidae is eventually reached, where all members share similar ecological roles as demersal predators, allowing classification into the demersal carnivore functional group. This hierarchical approach significantly reduces the number of required classifications, although is vulnerable to misclassifications at higher taxonomic levels if the LLM does not have sufficient ecological capability. The success of this approach is highly dependent on the quality of the grouping template and the LLM's ability to understand the ecological roles of taxa and is a key target for validation in this study. We provide an initial evaluation of the quality of this LLM-generated grouping in Section~\ref{supp:3}.


At each taxonomic level, the LLM evaluates taxa against the selected grouping template using the following prompt (where square brackets indicate dynamically updated variables):

\begin{prompt}
You are classifying marine organisms into functional groups for an Ecopath with Ecosim (EwE) model. Functional groups can be individual species or groups of species that perform a similar function in the ecosystem, i.e.\ have approximately the same growth rates, consumption rates, diets, habitats, and predators. They should be based on species that occupy similar niches, rather than of similar taxonomic groups.

Examine these taxa at the [rank] level and assign each to an ecological functional group.

Rules for assignment:
\begin{itemize}
\item If a taxon contains members with different feeding strategies or trophic levels, assign it to `RESOLVE'
\item Examples requiring `RESOLVE':
  \begin{itemize}
  \item A phylum containing both filter feeders and predators
  \item An order with both herbivores and carnivores
  \item A class with species across multiple trophic levels
  \end{itemize}
\item If all members of a taxon share similar ecological roles, assign to an appropriate group
\item Only consider the adult phase of the organisms, larvae and juveniles will be organized separately
\item Only assign a definite group if you are confident ALL members of that taxon belong to that group
\end{itemize}

Taxa to classify:
[List of taxa]

Available ecological groups (name: description):
[List of available groups and their descriptions]

Return only a JSON object with taxa as keys and assigned groups as values.
\end{prompt}

When the research focus indicates areas requiring higher resolution (e.g., commercial fisheries species), we modify the classification process with additional guidance:

\begin{prompt}
Special consideration for research focus:
The model's research focus is: [research focus]

When classifying taxa that are related to this research focus:
\begin{itemize}
\item Consider creating more detailed, finer resolution groupings
\item Keep species of particular interest as individual functional groups
\item For taxa that interact significantly with the focal species/groups, maintain higher resolution groupings
\item For other taxa, broader functional groups may be appropriate
\end{itemize}
\end{prompt}

The framework maintains complete provenance information, including the source of group definitions and any AI-suggested modifications. The system automatically includes a Detritus functional group to represent non-living organic matter in the ecosystem. Finally, a detailed grouping report is produced which documents all of the classification decisions for later human review.
