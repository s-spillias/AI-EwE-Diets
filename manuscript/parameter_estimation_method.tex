\subsubsection{Parameter Estimation (Likely to be removed - no validation)}

The final stage estimates the required Ecopath \citep{Christensen2004} parameters for each functional group. We query the EcoBase repository \citep{Colleter2015} using a structured search strategy that combines group names with regional context (e.g., "Western Australian shelf species in [group]"). For each functional group, we retrieve five key parameters from comparable models: habitat area fractions, biomass densities (t/km$^2$), production/biomass ratios (P/B, year$^{-1}$), consumption/biomass ratios (Q/B, year$^{-1}$), and ecotrophic efficiency (EE).

We implement parameter-specific validation checks. For biomass densities, we verify values fall within expected ranges for each functional group type. For P/B and Q/B ratios, we check consistency with known allometric relationships and life history characteristics. EE values must fall between 0 and 1, with additional validation against typical ranges for similar species groups.

Claude \citep{Anthropic2024} analyzes these parameter sets using standardized criteria. The analysis considers the relevance of each model to the group in question, the consistency of values across different models, trends or patterns in the data that might indicate suitable values, and the ecological context provided by model metadata.

For each parameter assignment, Claude provides a structured response containing the suggested parameter value, detailed reasoning for the selection, references to source models, and any assumptions or caveats.

When EcoBase data is insufficient or parameter values show high variability, we employ a hierarchical fallback system. First, we search for parameters from taxonomically similar groups. Next, we apply empirical relationships where available. Finally, if automated methods are unsuitable, we flag the parameter for manual parameterization.

The system maintains comprehensive logging of parameter assignments. This includes search queries and results, parameter validation checks, AI reasoning and decisions, source model references, and data quality assessments.

The final output consists of a structured JSON file containing the complete parameter set for each functional group. This format preserves the full provenance of each parameter value while enabling automated validation and mass-balance calculations in Ecopath.
