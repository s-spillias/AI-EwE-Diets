\begin{table}[htbp]
\centering
\caption{Dominant patterns of species classification instability across three study regions. The table presents the most frequent oscillation patterns between functional groups for species that were inconsistently classified across the five framework iterations. For each region, the total number of variably classified species is shown (representing less than 1.1\% of all species), along with the percentage distribution of different oscillation patterns.}
\label{tab:unstable_species}
\small
\begin{tabular}{llcc}
\hline
Region & Most Common Pattern & Count & \% of Total \\
\hline
Northern & Macrozoobenthos $\leftrightarrow$ Benthic infaunal carnivores & 28 & 27.2\% \\
Territory & Benthic filter feeders $\leftrightarrow$ Deposit feeders & 25 & 24.3\% \\
(103 species) & Prawns $\leftrightarrow$ Macrozoobenthos & 21 & 20.4\% \\
& Other patterns & 29 & 28.1\% \\
\hline
South East & Piscivores $\leftrightarrow$ Deep demersal fish & 42 & 33.6\% \\
Inshore & Benthic grazers $\leftrightarrow$ Benthic carnivores & 31 & 24.8\% \\
(125 species) & Planktivores $\leftrightarrow$ Mesopelagic fish & 28 & 22.4\% \\
& Other patterns & 24 & 19.2\% \\
\hline
South East & Benthic filter feeders $\leftrightarrow$ Benthic carnivores & 25 & 28.7\% \\
Offshore & Macrozoobenthos $\leftrightarrow$ Deep demersal fish & 22 & 25.3\% \\
(87 species) & Mesozooplankton $\leftrightarrow$ Macrozoobenthos & 18 & 20.7\% \\
& Other patterns & 22 & 25.3\% \\
\hline
\multicolumn{4}{p{0.95\textwidth}}{\small \textit{Note:} Arrows indicate group assignment oscillation between iterations. Complete species-level data available in Section S3 of the supplementary material.} \\
\hline
\end{tabular}
\end{table}
