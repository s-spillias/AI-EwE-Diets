\section{Discussion}

We have shown that an AI-driven framework is able to construct an important component, the diet matrix, of common ecosystem models, with a fair degree of reliability. This capability addresses a significant challenge in ecosystem modelling, as the increasing use of these models for environmental management and policy decisions requires efficient and accurate development approaches \citep{weiskopf2022increasing, schuwirth2019make}. Constructing reliable ecosystem models traditionally involves complex technical challenges, including species identification, data harvesting, and the creation of accurate diet matrices—processes that are particularly demanding in fisheries management contexts where food web models inform ecological and socioeconomic decision-making \citep{chakravorty2024systematic}. Our framework provides a systematic, AI-assisted solution that enhances reproducibility and dramatically reduces the time investment required from months to hours. This efficiency gain aligns with the growing recognition that effective ecological models must balance mechanistic understanding, appropriate spatial and temporal resolution, and uncertainty quantification to support decision-making \citep{schuwirth2019make}. By streamlining technical aspects through integration of multiple data sources \citep{Christensen2004, Colleter2015} and AI-driven synthesis \citep{spillias2024human, Noleto2024}, our approach allows modelers to dedicate more resources to stakeholder engagement and result communication, potentially increasing the impact of ecosystem modelling on environmental management.

\subsection{Validation Assessment}

Here we have demonstrated the framework's potential to generate reproducible, ecologically meaningful components for ecosystem model development while significantly reducing development time. It can complement the traditional approach to model building and expert judgement. The framework's ability to construct reliable ecosystem model components reveal both strengths and limitations. The framework demonstrates strong internal consistency, with high stability scores (98.8-99.6\%) in species classifications across regions and robust correlations in predator-prey rankings ($\rho = 0.72-0.89$). This consistency suggests the framework makes systematic rather than arbitrary decisions in constructing ecological relationships. However, these metrics must be interpreted cautiously, as they reflect the framework's reproducibility rather than ecological accuracy.

Our performance metrics are comparable to other recent AI applications in food web modelling, such as FoodWebAI \citep{Noleto2024}, which achieved 90-98\% accuracy in correctly determining species' positions in the food chain (trophic levels) and 74\% accuracy in identifying predator-prey relationships (trophic links) across three ecosystems. The grouping accuracy we found was less than others have reported for bespoke chatbots but higher than has been reported for frontier LLMs of the previous generation \citep{sahu4946942taxobot}. 

The Great Australian Bight comparison against expert knowledge provides important insights into the framework's reliability. The high success rate in identifying absent trophic interactions (73.1\%) indicates the framework effectively avoids spurious ecological connections. However, the moderate Kappa coefficient (0.38) with expert-assigned diet proportions and the tendency to miss specialised ecological groups reveals important limitations. The framework shows a bias toward generalised classifications that may overlook management-relevant distinctions, particularly when it comes to accommodating functional groups that are comprised of only a single species.

Our detailed taxonomic validation analysis further illuminates the framework's ecological accuracy, with 75.3\% of taxonomic assignments being fully correct and an additional 17.3\% being partially correct. This high overall accuracy rate (92.6\% at least partially correct) is consistent with other analyses \citep{Noleto2024,dorm2025large} and suggests the framework generally makes ecologically sound grouping decisions. However, the analysis also revealed that the success of the grouping may depend on the comprehensiveness of the grouping template that the AI system is provided. Further, deep-sea organisms, parasitic taxa, and species with complex or variable feeding strategies were more frequently misclassified or only partially correctly assigned. These findings highlight areas where the framework's ecological knowledge may be limited or where the predefined functional group templates may not adequately capture the full range of ecological roles present in marine ecosystems. It also highlights where human triage will be needed when using these methods, showing the kinds of groups where the modeller needs to focus checking or redefine model structures to reduce erroneous assignments. However, the uncertainty exhibited by the framework in classifying certain taxa, particularly deep-sea organisms and species with complex feeding strategies, should not be viewed solely as a limitation but also as a feature that appropriately reflects genuine scientific uncertainty. When knowledge about certain ecological niches is limited, a high level of certainty in classifications would be misleading. The framework's variable classifications and partial assignments for these challenging taxa actually represent a more honest representation of our current ecological understanding, aligning with principles of scientific transparency in uncertainty communication.

The framework's handling of ecological complexity shows mixed results. While it successfully captures broad trophic patterns and adapts to regional differences, its treatment of species that span multiple functional roles needs improvement. For instance, the variable classification of anemones and flatfishes between functional groups, while partially reflecting natural ecological flexibility, suggests the need for more nuanced classification approaches. Whilst here we have relied on the LLM's `embedded' ecological knowledge to classify species, providing additional ecological information from online databases may improve the quality of grouping assignments. Or perhaps, with the rapid rate of LLM capacity improvement, future LLM's will perform better at ecological tasks such as these. The identification of additional trophic interactions not present in expert matrices (14.5\%) requires careful evaluation - these could represent either over-connection or potentially valid relationships that merit further investigation.

These validation outcomes suggest the framework can serve as a useful starting point for ecosystem model development, particularly in its ability to avoid implausible ecological connections and maintain consistent broad-scale trophic structures. However, its outputs require expert review, especially for specialised ecological groups and complex trophic relationships or where there is only a qualitative understanding of ecoystem function. The balance between the framework's systematic approach and the need for ecological expertise emerges as a key consideration for its practical application.

\subsection{Applications for EBFM}

Given the validation outcomes, the framework shows promise as a rapid prototyping tool for model developement for ecosystem-based fisheries management. Its demonstrated ability to avoid spurious ecological connections while maintaining consistent broad-scale trophic structures makes it valuable for accelerating initial model development, reducing construction time from months to hours, with total processing times ranging from 2.6 to 4.9 hours per region. However, the framework's limitations with specialised groups and single-species functional units means it should be used as a starting and accelerator for expert refinement rather than a standalone solution. Further studies that explore the impact of prompt engineering and template choice on the quality of outcomes for a given context will help further demonstrate the frameworks capabilities and limitations.

The framework's systematic approach to uncertainty quantification helps identify where additional data collection or expert input is most needed. For instance, its higher performance in identifying absent interactions versus capturing expert-identified relationships suggests where manual review should focus. This aligns with approaches to uncertainty-aware ecosystem management \citep{hill2007model,link2012dealing}, while the framework's consistent methodology across regions supports standardised approaches to model development across jurisdictions.

\subsection{Limitations and Uncertainties}

Our validation results highlight three key limitations of the framework. First, the framework's bias toward generalised classifications, evidenced by its difficulty with single-species functional groups in the GAB comparison, reflects fundamental limitations in how the AI system processes ecological relationships. Second, the framework's reliance on Claude 3.5 Sonnet, a closed-source large language model, introduces scientific reproducibility challenges. While our validation demonstrates consistent performance, we cannot fully examine the model's decision-making process or potential biases. Third, practical implementation faces computational and data-related constraints. Data harvesting operations proved time-intensive, and the framework's performance varied with data availability across regions and ecological roles. Future iterations might benefit from exploring open-source alternatives \citep{Kommineni2024} and developing more transparent decision-making processes. Additionally, fishery and ecosystem managers need to trust that AI or hybrid approaches can reliably construct and parameterize models, requiring careful attention to risk, uncertainty, and transparency in model development. Decision makers often perceive ecological models as ``black boxes'' with questionable data inputs \citep{boschetti2018decision}, which may be amplified with AI-based approaches.


\subsection{Future Development and Assessment}

To address the identified limitations, several key areas require further development. First, the framework's handling of specialised ecological groups needs improvement, particularly for commercially important single-species units. This could involve developing more sophisticated protocols for identifying and preserving management-relevant distinctions during the grouping process, as different key groupings are more important for fishery management vs. spatial planning, for example. Second, to enhance scientific reproducibility, future versions should explore the capability of other LLMs, including open-source LLMs which can be more transparently assessed than proprietary models like Claude.

Third, systematic validation across diverse ecosystem types is needed to establish operational boundaries. This validation should encompass a range of ecosystems with varying structures, biodiversity levels, and data availability—including polar regions, coral reefs, deep ocean habitats, pelagic systems, and upwelling zones—and across multiple spatial scales from ocean basins to coastal bays. Testing should pay particular attention to how the framework handles specialised ecological roles in different contexts. Throughout this development process, the framework should maintain its role as a collaborative tool that complements rather than replaces expert judgment, focusing on rapid prototyping while preserving the critical role of ecological expertise in model refinement.

Fourth, we rely heavily on online databases which are subject to data quality issues and biases. Future work should explore how to incorporate local knowledge and expert judgment into the framework to address these limitations. This could involve developing more sophisticated data integration methods that combine structured data from online sources with unstructured local knowledge, as well as exploring how to incorporate expert feedback into the AI decision-making process - potentially involving weighting the importance of certain sources of data for the LLM. Additionally, testing whether using local or national databases (e.g., Fishes of Australia) alongside global databases provides relevant information missed in global databases would be valuable to determine if region-specific data sources can improve the framework's ecological accuracy. Current movements towards FAIR data principles \citep{tanhua2019ocean} will likely also improve the ability for AI systems to find the most accurate and relevant data sources.

Fifth, future development should incorporate established best practices for ecological model building to enhance quality and reliability. The Ecopath with Ecosim approach offers methodological standards \citep{Christensen2004, heymans2016best} and pre-balance diagnostics \citep{link2010adding} that could strengthen AI-assisted frameworks. More broadly, Good Modelling Practice (GMP) principles \citep{jakeman2024towards} emphasize the importance of explicating modelling choices throughout the entire modelling lifecycle to build trust in model insights within their social and political contexts. Integrating these established practices would improve model assessment, facilitate evaluation by management bodies, and help normalize rigorous standards across the modelling community; particularly important as ecological network models increasingly inform resource management decisions.

Finally, the framework's utility for ecosystem-based fisheries management should be further explored through case studies that evaluate its effectiveness in building models that support management decisions. This could involve comparing the performance of models built using the framework to those built using traditional methods, as well as assessing the framework's ability to support management-relevant analyses such as scenario testing and policy evaluation.

% \subsection{Conclusion}

% While we have shown some initial promising results in terms of the framework's ability to produce replicable diet matrices that share qualities with those used by expert human modellers, further assessment by experienced ecologists and EwE modellers is needed to evaluate the ecological soundness and utility of the produced groupings and diet matrices. The framework's diet proportion estimates should be compared against expert knowledge, focusing on both common and rare trophic relationships. These evaluations would establish clear boundaries for the framework's operational use in ecosystem modelling.

% We encourage ecologists to experiment with the framework as a collaborative tool rather than a replacement for expert judgment. The framework can rapidly generate initial model structures for expert review, enabling efficient iteration between computational suggestions and ecological expertise. Systematic testing across different ecosystem types, from data-rich temperate systems to data-sparse tropical environments, would establish the framework's utility across diverse ecological contexts. This collaborative approach would strengthen the bridge between artificial intelligence and ecological expertise in ecosystem modelling.
