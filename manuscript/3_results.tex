\section{Results}

Our validation framework assessed three key aspects of the AI-assisted ecosystem modelling approach: reproducibility of species groupings, consistency of diet matrix construction, and accuracy against expert-derived matrices. 

\subsection{Species Grouping Reproducibility}
% Addressing Objective 2a: Assess consistency of species grouping decisions

\subsubsection{Classification Consistency Analysis}
The framework successfully reduced ecological complexity while preserving meaningful biological relationships. Starting with 63 potential functional groups provided in the default template (See \ref{supp:technical_implementation}), it identified 34-37 region-specific groups. Chi-square tests confirmed the non-random nature of these groupings, showing consistent species assignments across all regions (p < 0.001). This statistical significance provides strong evidence that the framework makes systematic grouping decisions rather than arbitrary assignments.

The framework achieved high classification stability for groups across all regions. Mean consistency scores, where 1.0 represents identical species assignments to groups across all groups and iterations, were exceptionally high: 0.997 for both Northern Territory and South East shelf, and 0.998 for South East Offshore. This translated to very low proportions of species that were variably classified across the five iterations: only 0.99\% (103 species) in Northern Territory, 1.06\% (125 species) in South East shelf, and 0.73\% (87 species) in South East Offshore. These results demonstrate that the framework's classifications remained stable despite the stochastic nature of the AI decision-making process.

The Jaccard similarity indices reveal high overall stability in group membership across all three regions (Figure \ref{fig:regional_analysis}), with most functional groups showing indices above 0.95. The groups labelled in the figure represent notable outliers with lower stability indices, particularly among mobile and pelagic groups. Benthic groups consistently show the highest stability across all regions, while more mobile groups like piscivores and demersal fish demonstrate greater variability in their group membership.

Further detailed analysis of group stability patterns across regions is provided in \ref{supp:group_stability}. 

\begin{figure}[htbp]
    \centering
    \includegraphics[width=\textwidth]{figures/regional_group_analysis.png}
    \caption{Group membership stability across three regions measured by Jaccard similarity index (0.85-1.0). Most groups show high stability (>0.95), with labelled points indicating notable outliers that exhibit lower stability.}
    \label{fig:regional_analysis}
\end{figure}

\begin{table}[htbp]
\centering
\caption{Summary of Species Classification Instability Patterns}
\label{tab:unstable_species}
\small
\begin{tabular}{llcc}
\hline
Region & Most Common Pattern & Count & \% of Total \\
\hline
Northern & Macrozoobenthos $\leftrightarrow$ Benthic infaunal carnivores & 28 & 27.2\% \\
Territory & Benthic filter feeders $\leftrightarrow$ Deposit feeders & 25 & 24.3\% \\
(103 species) & Prawns $\leftrightarrow$ Macrozoobenthos & 21 & 20.4\% \\
& Other patterns & 29 & 28.1\% \\
\hline
South East & Piscivores $\leftrightarrow$ Deep demersal fish & 42 & 33.6\% \\
Inshore & Benthic grazers $\leftrightarrow$ Benthic carnivores & 31 & 24.8\% \\
(125 species) & Planktivores $\leftrightarrow$ Mesopelagic fish & 28 & 22.4\% \\
& Other patterns & 24 & 19.2\% \\
\hline
South East & Benthic filter feeders $\leftrightarrow$ Benthic carnivores & 25 & 28.7\% \\
Offshore & Macrozoobenthos $\leftrightarrow$ Deep demersal fish & 22 & 25.3\% \\
(87 species) & Mesozooplankton $\leftrightarrow$ Macrozoobenthos & 18 & 20.7\% \\
& Other patterns & 22 & 25.3\% \\
\hline
\multicolumn{4}{p{0.95\textwidth}}{\small \textit{Note:} Arrows indicate group assignment oscillation between iterations. Complete species-level data available in supplementary Table S1.} \\
\hline
\end{tabular}
\end{table}


\subsection{Diet Matrix Reproducibility}
% Addressing Objective 2b: Assess consistency of diet matrix values

\subsubsection{Trophic Interaction Consistency}
The framework identified consistent trophic relationships across all regions, with the Northern Territory showing 358 interactions (58.7\% having stability scores > 0.7), South East shelf 380 interactions (51.3\% stable), and South East Offshore 477 interactions (56.0\% stable). Spearman correlations between iterations demonstrated that the relative proportions of different prey in predator diets remained consistent across all regions (Northern Territory: $\rho = 0.72-0.89$; South East shelf: $\rho = 0.68-0.85$; South East Offshore: $\rho = 0.70-0.87$), even when absolute proportions varied. Detailed diet matrices for each region are provided in \ref{supp:diet_matrix}.

\begin{figure}[htbp]
    \centering
    \includegraphics[width=\textwidth]{figures/stability_score_distribution.png}
    \caption{Distribution of diet interaction stability scores across regions. Half-violin plots show the density of stability scores (1=stable, 0=unstable), with embedded box plots indicating quartiles and median. Individual points represent specific predator-prey interactions, and the red dashed line shows the mean stability score across all regions. The distributions are bounded at one, reflecting perfect stability, with most interactions showing scores above 0.7. Stability scores quantify the consistency of predator-prey interactions across iterations, where a score of 1.0 indicates the interaction was identified with identical diet proportions in all iterations, while lower scores reflect either variable diet proportions or inconsistent identification of the interaction. }
    \label{fig:stability_distribution}
\end{figure}

\subsection{Grouping and Diet Proportion Accuracy Assessment: Great Australian Bight Case Study}
% Addressing Objective 2c: Assess accuracy against expert-created matrices

To evaluate the framework's accuracy against expert knowledge, we compared its output to an expert-derived Ecopath model of the Great Australian Bight ecosystem. The framework demonstrated varying performance across functional groups, successfully matching 59 of 76 expert-defined groups (77.6\%). The framework omitted 17 groups present in the expert matrix, including several commercially important species (Southern Bluefin Tuna, Snapper, King George whiting) and specialised ecological groups (Abalone, Nanozooplankton). Conversely, it generated only 3 groups not present in the expert matrix.

As shown in Figure \ref{fig:gab_comparison}a, the framework demonstrated varying levels of agreement across different functional groups in identifying trophic interactions. The analysis revealed that 15\% of interactions were present in both matrices (dark purple), while 70\% were correctly identified as absent in both (light grey). The framework uniquely identified 10\% of interactions (teal) that were not present in the expert matrix, while missing 5\% of expert-identified interactions (yellow). This pattern varied substantially across functional groups, with higher trophic levels showing greater discrepancies between AI and expert matrices.

The framework showed moderate success in capturing the quantitative aspects of diet proportions. The distribution of absolute differences in diet proportions (Figure \ref{fig:gab_comparison}b) revealed that the majority of differences were relatively small, with approximately 80\% of the differences being less than 0.2. However, a long tail in the distribution indicates some cases where AI-generated proportions diverged substantially from expert values. This suggests that when the framework correctly identified a trophic interaction, it often estimated diet proportions within reasonable bounds of expert values, though with notable variations across different predator-prey combinations. A comprehensive visualization of these differences across all functional groups is provided in \ref{supp:diet_matrix}.

\begin{landscape}
\begin{figure}[htbp]
    \centering
    \includegraphics[width=1.2\paperwidth]{figures/diet_matrix_validation/simplified_comparison.png}
    \caption{Comparison of expert-created and AI-generated diet matrices for the Great Australian Bight ecosystem. (a) Presence-absence patterns showing the proportion of different interaction types across functional groups. Dark purple indicates interactions present in both matrices, yellow shows expert-only interactions, teal shows AI-only interactions, and light grey indicates absence in both. (b) Distribution of absolute differences in diet proportions where both matrices indicate an interaction, showing the frequency of different magnitudes of disagreement between AI and expert estimates.}
    \label{fig:gab_comparison}
\end{figure}
\end{landscape}

\subsection{Framework Implementation and Performance}
% Addressing Objective 1: Present a systematic, AI-assisted framework for assembling and parameterizing EwE diet matrices

\subsubsection{Scale and Processing Efficiency}
We evaluated our framework through five independent runs across three distinct Australian regions, processing a total of 39,722 species. The framework handled 10,621 species in the Northern Territory's tropical reef ecosystem, 17,068 in the South East shelf's coastal and pelagic environments, and 12,033 in the South East Offshore's deep-water systems. 

\subsubsection{Computational Efficiency}
The computational requirements of the AI framework varied across regions. Total processing time ranged from 2.8 to 4.8 hours across regions. The most time-intensive stage was the downloading of biological data from online databases, accounting for approximately 70\% of the total processing time. Species identification typically required 0.01 hours, while the AI-driven species grouping process averaged 0.26 hours. Diet data collection and matrix construction required 0.7 and 0.04 hours respectively, with final parameter estimation taking 0.20 hours. On average, the framework required 0.7 seconds per species for data downloading and 0.2 seconds per species for diet data collection, though these rates varied considerably between regions due to differences in data availability and species complexity.
\begin{table}[htbp]
\centering
\footnotesize
\caption{Computational requirements by region and processing stage}
\label{tab:timing_analysis}
\begin{tabular}{lccccccc}
\hline
Region & Species & \multicolumn{6}{c}{Processing Time (hours)} \\
\cline{3-8}
 & Count & Identification & Data & Grouping & Diet & Matrix & Parameter \\
 & & & Download & & Collection & Construction & Estimation \\
\hline
v2 NorthernTerritory & 11,362 & 0.01 & 2.2 & 0.2 & 0.2 & 0.04 & 0.2 \\
v2 SouthEastInshore & 13,901 & 0.01 & 2.8 & 0.2 & 1.6 & 0.04 & 0.2 \\
v2 SouthEastOffshore & 15,821 & 0.01 & 3.3 & 0.4 & 0.3 & 0.04 & --- \\
\hline
\end{tabular}
\vspace{1ex}
\end{table}


Processing times varied by region and stage (Table \ref{tab:timing_analysis}). Data harvesting required 2.2 hours for the Northern Territory and 2.8 hours for the South East shelf, with South East Offshore requiring 3.3 hours. Diet data collection took 0.2 hours for the Northern Territory and 1.6 hours for the South East shelf, with South East Offshore requiring 0.3 hours. Species identification remained consistent at 0.01 hours across all regions, while grouping varied slightly from 0.2 hours for Northern Territory and South East shelf to 0.4 hours for South East Offshore.
