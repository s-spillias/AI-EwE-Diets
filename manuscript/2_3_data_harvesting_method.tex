\subsubsection{Data Harvesting}

Following species identification, the framework gathers ecological and life history information for each identified species. From SeaLifeBase and FishBase \citep{froese2010fishbase}, it extracts a range of information, including habitat preferences (marine, brackish, or freshwater), depth range distributions, maximum body lengths, and diet data. These databases are accessed through their publicly available PARQUET files using DuckDB for efficient querying of large datasets. 

Our species filtering protocol implements specific constraints to align with Ecopath with Ecosim modelling requirements. When processing taxonomic data, we prioritize species-level entries over genus-level classifications, only defaulting to genus-level when species-specific data is unavailable. For diet composition data, we extract data on food items including prey species codes ('SpecCode'), food groups, and prey stages (as is found in FishBase and SeaLifeBase; see [Github repository for specific queries to these databases]). We specifically filter for juvenile and/or adult life stages, as larval stages are typically incorporated into planktonic functional groups rather than treated as separate components of adult diets.

We supplement the base biological data with interaction information from the Global Biotic Interactions (GLOBI) database \citep{Poelen2014}. For each species, we query the GLOBI API using URL-encoded species names to retrieve interaction records in CSV format. The GLOBI data processing preserves the raw interaction data and treats directional relationships ('eats'/'preysOn' and 'eatenBy'/'preyedUponBy') as complementary evidence of trophic interactions. For each predator-prey group pair, we tally the total number of observed interactions, which provides information about the relative frequency of feeding relationships between groups. We further enrich this data through retrieval-augmented generation (RAG) searches of regional literature (detailed in Section~\ref{supp:4.2} of the supplementary material), focusing on specific feeding relationships and dietary preferences.

Technical implementation details are provided in Section~\ref{supp:1} of the supplementary material.
