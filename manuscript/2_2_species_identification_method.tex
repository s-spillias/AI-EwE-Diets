\subsubsection{Species Identification}

The framework begins by taking a user-defined shape file that defines the study region boundaries. It then accesses the Ocean Biodiversity Information System \citep{Grassle1999} through the \texttt{robis} R package \citep{robis}, which enables automated querying and data processing. We chose OBIS as our primary data source due to its extensive marine species coverage and standardized taxonomic classifications. 

The framework uses the `checklist' function from \texttt{robis} to retrieve scientific names and complete taxonomic classifications from kingdom to species level for all recorded species within these boundaries. Whilst collecting all occurrences would help with estimating distributions and biomasses for other modelling purposes, the time and computational resources required to process such large datasets are prohibitive and so we focus on presence only. 

To limit the amount of processing required by the LLM, the raw OBIS data is transformed in two steps. First, it filters the dataset using OBIS's \texttt{is\_marine} flag to eliminate terrestrial species that may occur in coastal records. Second, it removes taxonomic redundancy using a rank-based approach that retains only the most specific classification level available. For example, if our dataset contains both \textit{Chrysophrys auratus} (species-level) and \textit{Chrysophrys} (genus-level) entries for the same organism, our algorithm retains only the species-level entry. Our approach processes taxonomic ranks from most specific (scientific name) to most general (kingdom), keeping only the entry with the highest taxonomic resolution for each organism. 

The final species list is stored in a structured CSV file containing verified marine species and their complete taxonomic hierarchies. The complete R implementation, including rank-based filtering algorithms and geographic processing functions, is available in the project repository.
